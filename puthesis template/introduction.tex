%
%  revised  introduction.tex  2011-09-02  Mark Senn  http://engineering.purdue.edu/~mark
%  created  introduction.tex  2002-06-03  Mark Senn  http://engineering.purdue.edu/~mark
%
%  This is the introduction chapter for a simple, example thesis.
%


\chapter{Introduction}

Following a brief introductory paragraph, the introduction chapter requires six headings titled the following: Problem Topic, Background, Criteria and Parameter Restrictions, Methodology, Primary Purpose, and Overview. Each heading should be margin flush, bold and underlined (APA level 3 heading). A description of what is to be written under each heading is indicated below.

\section{Problem Topic}

Under the heading, Problem Topic, bring into focus in a single clear statement, the exact problem addressed and the nature of the end result sought. The problem itself must not be stated in terms of a desirable goal or solution but rather as a current or potential negative situation, result, etc. that the employer would like to eliminate or avoid. If sub-problems are involved, they should be stated here.

\section{Background}

Under the heading, Background, provide a helpful orientation regarding the problem addressed by the thesis.  It may be necessary to clarify the situation in which the problem arose.  The nature of the student's experience with that situation might also be appropriate information to include.  Concerns expressed by managers, customers, or others may be relevant background.  All the material presented should enable the reader to understand the exact nature of the problem and is importance especially to the employer organization.  The background should not give elaborate overly-generalized, obvious, or  nonessential information.  Facts that should be reporte din a later chapter should not be included.  If a history of the employer organization is relevant, it should be placed in an appendix.

\section{Criteria and Parameter Restrictions}

Under the heading Criteria and Parameter Restrictions, identify the criteria and parameters imposed, if the thesis project concerns the creation of a design.  (Sometimes parameters are also briefly identified earlier under the heading Problem Topic, as part of the problem or the problem situation.)  If the thesis prject does not concern the creation of a design, the heading may simply be Criteria or perhaps Standards.  For all topics, criteria or standards should be identified as the basis for judging the data later presented in supporting (developing) chapters.  The appropriateness of each criterion or standard must be briefly justified.
\subsection{Subsection heading}

This is a sentence.
This is a sentence.
This is a sentence.
This is a sentence.
This is a sentence.


\subsubsection{Subsubsection heading}

This is a sentence.
This is a sentence.
This is a sentence.
This is a sentence.
This is a sentence.
